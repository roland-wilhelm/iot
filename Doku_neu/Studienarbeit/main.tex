% Dokumentenkopf ---------------------------------------------------------------
%   Diese Vorlage basiert auf "scrreprt" aus dem koma-script.
% ------------------------------------------------------------------------------
\documentclass[
	ngerman,
    11pt, % Schriftgr??e
    DIV10, % Seitengroesse (siehe Koma Skript Dokumentation !)
    ngerman, % f?r Umlaute, Silbentrennung etc.
    a4paper, % Papierformat
    oneside, % einseitiges Dokument
    titlepage, % es wird eine Titelseite verwendet
    parskip=half, % Abstand zwischen Abs?tzen (halbe Zeile)
    headings=normal, % Gr??e der ?berschriften verkleinern
%	chapterprefix=true, % Ausgabe Kapitel + Nummer
    listof=totoc, % Verzeichnisse im Inhaltsverzeichnis auff?hren
	bibliography=totoc, % Literaturverzeichnis im Inhaltsverzeichnis auff?hren
    index=totoc, % Index im Inhaltsverzeichnis auff?hren
    captions=tableheading, % Beschriftung von Tabellen unterhalb ausgeben
    final, % Status des Dokuments (final/draft)
	pointlessnumbers % keine Nummern hinter Kapitelnummern
%	numbers=noenddot
	abstracton % Erzeugen einer Zusammenfassung
]{scrreprt}


% Meta-Informationen -----------------------------------------------------------
%   Informationen ?ber das Dokument, wie z.B. Titel, Autor, Matrikelnr. etc
%   werden in der Datei Meta.tex definiert und k?nnen danach global
%   verwendet werden.
% ------------------------------------------------------------------------------
% Meta-Informationen -----------------------------------------------------------
%   Definition von globalen Parametern, die im gesamten Dokument verwendet
%   werden k�nnen (z.B auf dem Deckblatt etc.).
%
%   ACHTUNG: Wenn die Texte Umlaute oder ein Esszet enthalten, muss der folgende
%            Befehl bereits an dieser Stelle aktiviert werden:
%            \usepackage[latin1]{inputenc}
% ------------------------------------------------------------------------------

\newcommand{\titel}{Praktikum Embedded- und Echtzeitbetriebssysteme}
\newcommand{\titeleng}{xy}
\newcommand{\untertitel}{}
\newcommand{\art}{Studienarbeit}
\newcommand{\fachgebiet}{Fakult�t f�r Informatik und Mathematik}
\newcommand{\autor}{Roland Wilhelm}
\newcommand{\studienbereich}{Embedded Computing}
\newcommand{\matrikelnr}{12 34 56}
\newcommand{\erstgutachter}{xy}
\newcommand{\zweitgutachter}{xyz}
\newcommand{\jahr}{2012}
\newcommand{\ort}{Hochschule M�nchen}
\newcommand{\logo}{Bilder/hm.png}


% ben?tigte Packages -----------------------------------------------------------
%   LaTeX-Packages, die ben?tigt werden, sind in die Datei Packages.tex
%   "ausgelagert", um diese Vorlage m?glichst ?bersichtlich zu halten.
% ------------------------------------------------------------------------------
% Anpassung des Seitenlayouts --------------------------------------------------
%   siehe Seitenstil.tex
% ------------------------------------------------------------------------------
\usepackage[
    automark, % Kapitelangaben in Kopfzeile automatisch erstellen
    headsepline, % Trennlinie unter Kopfzeile
    ilines % Trennlinie linksbündig ausrichten
]{scrpage2}

% Anpassung an Landessprache ---------------------------------------------------
\usepackage[ngerman]{babel}

% Umlaute ----------------------------------------------------------------------
%   Umlaute/Sonderzeichen wie äüöß direkt im Quelltext verwenden (CodePage).
%   Erlaubt automatische Trennung von Worten mit Umlauten.
% ------------------------------------------------------------------------------
%\usepackage[latin1]{inputenc}
\usepackage[utf8]{inputenc}
\usepackage[T1]{fontenc}
\usepackage{textcomp} % Euro-Zeichen etc.


% Schrift ----------------------------------------------------------------------
\usepackage{lmodern} % bessere Fonts
\usepackage{relsize} % Schriftgröße relativ festlegen

% Grafiken ---------------------------------------------------------------------
% Einbinden von JPG-Grafiken ermöglichen
\usepackage[dvips,final]{graphicx}
% hier liegen die Bilder des Dokuments
\graphicspath{{Bilder/}}

% Befehle aus AMSTeX für mathematische Symbole z.B. \boldsymbol \mathbb --------
\usepackage{amsmath,amsfonts}

% für Index-Ausgabe mit \printindex --------------------------------------------
\usepackage{makeidx}

% Einfache Definition der Zeilenabstände und Seitenränder etc. -----------------
\usepackage{setspace}
\usepackage{geometry}

% Einfaches Packet um ganze Blöcke als Kommentar zu kennzeichnen z.b. \begin{comment} ... \end{comment}
\usepackage{comment}


% ermöglicht das einbinden von pdf Dateien oder nur bestimmte Seiten
\usepackage{pdfpages}

% Symbolverzeichnis ------------------------------------------------------------
%   Symbolverzeichnisse bequem erstellen. Beruht auf MakeIndex:
%     makeindex.exe %Name%.nlo -s nomencl.ist -o %Name%.nls
%   erzeugt dann das Verzeichnis. Dieser Befehl kann z.B. im TeXnicCenter
%   als Postprozessor eingetragen werden, damit er nicht ständig manuell
%   ausgeführt werden muss.
%   Die Definitionen sind ausgegliedert in die Datei "Glossar.tex".
% ------------------------------------------------------------------------------
%\usepackage[intoc]{nomencl}
%\let\abbrev\nomenclature
%\renewcommand{\nomname}{Abkürzungsverzeichnis}
%\setlength{\nomlabelwidth}{.25\hsize}
%\renewcommand{\nomlabel}[1]{#1 \dotfill}
%\setlength{\nomitemsep}{-\parsep}

% Abkürzungsverzeichnis, Abkürzungen werden nur angezeigt wenn diese im Text auch benutzt werden
\usepackage[printonlyused]{acronym}


% zum Umfließen von Bildern ----------------------------------------------------
\usepackage{floatflt}


% zum Einbinden von Programmcode -----------------------------------------------
\usepackage{listings}
\usepackage{xcolor} 
\definecolor{hellgelb}{rgb}{1,1,0.9}
\definecolor{colKeys}{rgb}{0,0,1}
\definecolor{colIdentifier}{rgb}{0,0,0}
\definecolor{colComments}{rgb}{1,0,0}
\definecolor{colString}{rgb}{0,0.5,0}
\lstset{
    float=hbp,
    basicstyle=\ttfamily\color{black}\small\smaller,
    identifierstyle=\color{colIdentifier},
    keywordstyle=\color{colKeys},
    stringstyle=\color{colString},
    commentstyle=\color{colComments},
    columns=flexible,
    tabsize=2,
    frame=single,
    extendedchars=true,
    showspaces=false,
    showstringspaces=false,
    numbers=left,
    numberstyle=\tiny,
    breaklines=true,
    backgroundcolor=\color{hellgelb},
    breakautoindent=true
}

% URL verlinken, lange URLs umbrechen etc. -------------------------------------
\usepackage{url}

% wichtig für korrekte Zitierweise ---------------------------------------------
\usepackage[square,numbers]{natbib}

% PDF-Optionen -----------------------------------------------------------------
\usepackage[
    bookmarks,
    bookmarksopen=true,
    colorlinks=true,
% diese Farbdefinitionen zeichnen Links im PDF farblich aus
    linkcolor=black, % einfache interne Verknüpfungen, default:red
    anchorcolor=black,% Ankertext, default:black
    citecolor=black, % Verweise auf Literaturverzeichniseinträge im Text, default:blue
    filecolor=black, % Verknüpfungen, die lokale Dateien öffnen, default:magenta
    menucolor=black, % Acrobat-Menüpunkte, default:red
    urlcolor=black, 
% diese Farbdefinitionen sollten für den Druck verwendet werden (alles schwarz)
    %linkcolor=black, % einfache interne Verknüpfungen
    %anchorcolor=black, % Ankertext
    %citecolor=black, % Verweise auf Literaturverzeichniseinträge im Text
    %filecolor=black, % Verknüpfungen, die lokale Dateien öffnen
    %menucolor=black, % Acrobat-Menüpunkte
    %urlcolor=black, 
    backref,
    plainpages=false, % zur korrekten Erstellung der Bookmarks
    pdfpagelabels, % zur korrekten Erstellung der Bookmarks
    hypertexnames=false, % zur korrekten Erstellung der Bookmarks
    linktocpage % Seitenzahlen anstatt Text im Inhaltsverzeichnis verlinken
]{hyperref}
% Befehle, die Umlaute ausgeben, führen zu Fehlern, wenn sie hyperref als Optionen übergeben werden
\hypersetup{
    pdftitle={\titel \untertitel},
    pdfauthor={\autor},
    pdfcreator={\autor},
    pdfsubject={\titel \untertitel},
    pdfkeywords={\titel \untertitel},
}



% fortlaufendes Durchnummerieren der Fußnoten ----------------------------------
\usepackage{chngcntr}

% für lange Tabellen -----------------------------------------------------------
\usepackage{longtable}
\usepackage{array}
\usepackage{ragged2e}
\usepackage{lscape}

% Spaltendefinition rechtsbündig mit definierter Breite ------------------------
\newcolumntype{w}[1]{>{\raggedleft\hspace{0pt}}p{#1}}

% Formatierung von Listen ändern -----------------------------------------------
\usepackage{paralist}

% bei der Definition eigener Befehle benötigt
\usepackage{ifthen}

% definiert u.a. die Befehle \todo und \listoftodos
\usepackage{todonotes}

% sorgt dafür, dass Leerzeichen hinter parameterlosen Makros nicht als Makroendezeichen interpretiert werden
\usepackage{xspace}


% Erstellung eines Index und Abk?rzungsverzeichnisses aktivieren ---------------
\makeindex
%\makenomenclature

% Kopf- und Fu?zeilen, Seitenr?nder etc. ---------------------------------------
% Zeilenabstand 1,5 Zeilen -----------------------------------------------------
\onehalfspacing

% Seitenränder -----------------------------------------------------------------
\setlength{\topskip}{\ht\strutbox} % behebt Warnung von geometry
\geometry{paper=a4paper,left=35mm,right=35mm,top=30mm}

% Kopf- und Fußzeilen ----------------------------------------------------------
\pagestyle{scrheadings}
% Kopf- und Fußzeile auch auf Kapitelanfangsseiten
%\renewcommand*{\chapterpagestyle}{scrheadings} 
% Schriftform der Kopfzeile
\renewcommand{\headfont}{\normalfont}

% Kopfzeile
\ihead{\textit{\headmark}}
\chead{}
\ohead{\pagemark}
\setlength{\headheight}{21mm} % Höhe der Kopfzeile
\setheadsepline[text]{0.4pt} % Trennlinie unter Kopfzeile

%\ihead{\large{\textsc{\titel}}\\ \small{\untertitel} \\[2ex] \textit{\headmark}}
%\ohead{\includegraphics[scale=0.15]{\logo}}
% Kopfzeile über den Text hinaus verbreitern
%\setheadwidth[0pt]{textwithmarginpar} 


% Fußzeile
\ifoot{}
\cfoot{}
\ofoot{}

% sonstige typographische Einstellungen ----------------------------------------

% erzeugt ein wenig mehr Platz hinter einem Punkt
\frenchspacing 

% Schusterjungen und Hurenkinder vermeiden
\clubpenalty = 10000
\widowpenalty = 10000 
\displaywidowpenalty = 10000

% Quellcode-Ausgabe formatieren
\lstset{numbers=left, numberstyle=\tiny, numbersep=5pt, breaklines=true}
\lstset{emph={square}, emphstyle=\color{red}, emph={[2]root,base}, emphstyle={[2]\color{blue}}}

% Fußnoten fortlaufend durchnummerieren
\counterwithout{footnote}{section}


% eigene Definitionen f?r Silbentrennung
% Trennvorschl�ge im Text werden mit \" angegeben
% untrennbare W�rter und Ausnahmen von der normalen Trennung k�nnen in dieser
% Datei mittels \hyphenation definiert werden

% eigene LaTeX-Befehle
% Eigene Befehle und typographische Auszeichnungen für diese

% einfaches Wechseln der Schrift, z.B.: \changefont{cmss}{sbc}{n}
\newcommand{\changefont}[3]{\fontfamily{#1} \fontseries{#2} \fontshape{#3} \selectfont}

% Abkürzungen mit korrektem Leerraum 
\newcommand{\ua}{\mbox{u.\,a.\ }}
\newcommand{\zb}{\mbox{z.\,B.\ }}
\newcommand{\dahe}{\mbox{d.\,h.\ }}
\newcommand{\vgl}{Vgl.\ }
\newcommand{\bzw}{bzw.\ }
\newcommand{\evtl}{evtl.\ }
\newcommand{\uvm}{\mbox{u.\,v.\,m.\ }}
%\newcommand{\dh}{\mbox{d.\,h.\ }}

\newcommand{\Abbildung}[1]{Abbildung~\ref{abb:#1}}

\newcommand{\bs}{$\backslash$}

% erzeugt ein Listenelement mit fetter Überschrift 
\newcommand{\itemd}[2]{\item{\textbf{#1}}\\{#2}}

% einige Befehle zum Zitieren --------------------------------------------------
\newcommand{\Zitat}[2][\empty]{\ifthenelse{\equal{#1}{\empty}}{\citep{#2}}{\citep[#1]{#2}}}

% zum Ausgeben von Autoren
\newcommand{\AutorName}[1]{\textsc{#1}}
\newcommand{\Autor}[1]{\AutorName{\citeauthor{#1}}}

% verschiedene Befehle um Wörter semantisch auszuzeichnen ----------------------
\newcommand{\NeuerBegriff}[1]{\textbf{#1}}
\newcommand{\Fachbegriff}[1]{\textit{#1}}

\newcommand{\Eingabe}[1]{\texttt{#1}}
\newcommand{\Code}[1]{\texttt{#1}}
\newcommand{\Datei}[1]{\texttt{#1}}
\newcommand{\Klasse}[1]{\texttt{#1}}
\newcommand{\Interface}[1]{\texttt{#1}}
\newcommand{\Komponente}[1]{\mbox{\glqq {#1}\grqq}}
\newcommand{\Anfuehrung}[1]{\mbox{\glqq {#1}\grqq}}


\newcommand{\Datentyp}[1]{\textsf{#1}}
\newcommand{\XMLElement}[1]{\textsf{#1}}
\newcommand{\Webservice}[1]{\textsf{#1}}


% Das eigentliche Dokument -----------------------------------------------------
%   Der eigentliche Inhalt des Dokuments beginnt hier. Die einzelnen Seiten
%   und Kapitel werden in eigene Dateien ausgelagert und hier nur inkludiert.
% ------------------------------------------------------------------------------
\begin{document}

% auch subsubsection nummerieren
\setcounter{secnumdepth}{3}
\setcounter{tocdepth}{3}

% Deckblatt und Abstract ohne Seitenzahl
\ohead{}
\begin{titlepage}
	\begin{center}		
		\vspace*{1.5cm} % * forces the space, else the space will not be shown
		\vspace*{1cm}
		
		\begin{center}
			\includegraphics[width=0.90\textwidth]\logo
		\end{center}
		\vspace{1.0cm}
		{\Huge Vorlesungsfach\\}
		\vspace{1.0cm}
		{\Large Team}\\
		\vspace{0.2cm}
		{\Large Studienarbeit ...}\\
		\vspace{0.2cm}
		{\Large Roland Wilhelm}\\
		
		
		\vspace*{7.0cm}
		
		{\normalsize Dozent: ...} \\
		{\normalsize Hochschule M�nchen} \\
		{\normalsize Master Informatik} \\	
		{\normalsize \today}		
		
		\vspace{1.5cm}		
	
	  \vspace{2cm}		
	\end{center} 
\end{titlepage}

%\includegraphics[width=1.00\textwidth]{thema.pdf}
%\includegraphics[width=1.00\textwidth]{erklaerung.pdf}
%\includepdf{thema.pdf} 
% Selbst?ndigkeitserkl?rung
\ohead{\pagemark}

% Seitennummerierung -----------------------------------------------------------
%   Vor dem Hauptteil werden die Seiten in gro?en r?mischen Ziffern 
%   nummeriert.
% ------------------------------------------------------------------------------
\pagenumbering{Roman}
\begin{abstract}
Hier steht die Kurzfassung ...
\end{abstract}

% Inhaltverzeichnis im PDF anzeigen aber nicht im Inhaltverzeichnis selbst
%\clearpage %%% ggf. \cleardoublepage
%\phantomsection
\clearpage
\pdfbookmark{\contentsname}{toc}
\tableofcontents %die Tilden zum Einrücken im Bookmark

%\tableofcontents % Inhaltsverzeichnis
\clearpage
\listoffigures % Abbildungsverzeichnis
\clearpage
\listoftables % Tabellenverzeichnis
\clearpage
\renewcommand{\lstlistlistingname}{Listingsverzeichnis}
\lstlistoflistings % Listings-Verzeichnis
\clearpage

% Abkürzungsverzeichnis --------------------------------------------------------
\addsec{Abk�rzungsverzeichnis} %�berschrift im Inhaltsverueichnis auflisten aber ohne Nummerierung
\begin{acronym} %[Abk.] l�ngste form angeben zum ausrichten
%\ac{Abk.}         % f�gt die Abk�rzung ein, au�er beim ersten Aufruf, hier wird die Erkl�rung mit angef�gt
%\acs{Abk.}        % f�gt die Abk�rzung ein
%\acf{Abk.}        % f�gt die Abk�rzung UND die Erkl�rung ein
%\acl{Abk.}        % f�gt nur die Erkl�rung ein
%\acro{3GPP}{3rd Generation Partnership Project}	% Hinzuf�gen einer neuen Abk�rzung

\acro{IMS}{IP Multimedia Subsystem}
\acro{CSCF}{Call Session Control Function}
\acro{HSS}{Home Subscriber Server}
\acro{NGN}{Next Generation Network}
\acro{SOA}{Service Oriented Architecture}
\acro{SER}{SIP Express Router}
\acro{SIP}{Session Initiation Protocol}
\acro{FHoSS}{FOKUS Home Subscriber Server}
\acro{ISC}{IMS Service Control}

\end{acronym}

%\input{Inhalt/Glossar}
% f?r korrekte ?berschrift in der Kopfzeile
%\clearpage\markboth{\nomname}{\nomname} 
%\printnomenclature
%\label{sec:Glossar}


% arabische Seitenzahlen im Hauptteil ------------------------------------------
\clearpage
\pagenumbering{arabic}

% die Inhaltskapitel werden in "Inhalt.tex" inkludiert -------------------------
%\linespread{3.0}
% Hier k�nnen die einzelnen Kapitel inkludiert werden. Sie m�ssen in den 
% entsprechenden .TEX-Dateien vorliegen. Die Dateinamen k�nnen nat�rlich 
% angepasst werden.

\section{Einleitung}
\label{sec:Einleitung}
\section{Netzwerkinfrastruktur}
\label{sec:openims_netzwerk}

\section{Installation und Konfiguration}
\label{sec:openims_config}
\todo{Rechtschreibfehler in der ganzen Section}

\subsection{Die Komponenten}
Auf der fertig eingerichteten Infrastruktur wurden die verschiedenen IMS Komponenten nach den Schritten die im offiziellen Install Guide auf openimscore.org erklärt werden\footnote{\url{http://www.openimscore.org/installation_guide}} instaliert.
Da fertige Debian Pakete bereit gestellt wurden, entfielen die Schritte zwei und drei.

Die Komponenten Wurden wie folgt verteilt:

\subsubsection{Home Subscriber Server}
Der HSS ist auf der Maschiene \lstinline{hss} instaliert, ausserdem ein MySQL Server der für den Betrieb dess HSS benötigt wird und das neueste JDK.

\subsubsection{Weboberfläche}

Der HSS kann über eine Weboberfläche konfiguriert werden, welche auf dem Apache Tomcat läuft und auf eine weitere MySQL Datenbank erfordert.
Diese wurde ebenfalls auf der Maschine \lstinline{hss} installiert.

Wenn in der Datei  \lstinline{/opt/OpenIMSCore/FHoSS/config/hhss.properties} als \lstinline{isten-Interface} eine IPv6 Adresse eingetragen ist, wird eine \lstinline{javax.management.MalformedObjectNameException(INVALID CHARACTERS ":")} geworfen. Um dieses Problem zu umgehen, wurde die IPv4 Adresse eingetragen und eine neuer DNS Eintrag \lstinline{hss4} erstellt der auf die IPv4 Adresse des \lstinline{hss} zeigt.
Das Webinterface ist unter der Adresse \url{http://hss4:8080} erreichbar. 
Benutzer für das WebInterface werden in der Datei \lstinline{/opt/OpenIMSCore/FHoSS/tomcat/conf/tomcat-users.xml} verwaltet. Die folgenden sind bisher vorhanden:
\begin{description}
\item [hss] mit Passwort \lstinline{hss} hat die Rolle \lstinline{Benutzer}
\item [hssAdmin] mit dem Passwort \lstinline{hss} hat die Rolle \lstinline{Admin}
\end{description}
\todo{stimmt das noch}

Das Tool zur Konfiguration des HSS, \lstinline{dpkg-reconfigure openimscore-fhoss} darf nicht mehr verwendet werden, da es die Resolv Einstellungen verändert und die Weboberfläche dann nicht mehr erreichbar ist.

\subsection{Interrogation-CSCF}
Der Interrogator wurde auf der Maschiene \lstinline{icscf} installiert. Er benötigt für den Betrieb einen MySQL Server der lokal installiert wurde.
\lstinline{dpkg-reconfigure openimscore-icscf} kann für weitere Konfiguration verwendet werden.

\subsection{Server-CSCF}
Die Serving Komponente wurde auf dem \lstinline{scscf} installiert.
\lstinline{dpkg-reconfigure openimscore-scscf} kann für weitere Konfiguration verwendet werden.

\subsection{Proxy-CSCF}
Ein Proxy ist auf dem  \lstinline{pcscf} installiert. dabei gab es nur einen kleineren Fehler, im Debian Pakte hat die Datei \lstinline{/usr/lib/ser/emerg_info.xml} gefehlt, diese wurde aus dem Upstream Repository kopiert.
Das Tool \lstinline{dpkg-reconfigure openimscore-pcscf} kann zur weiteren Konfiguration verwendet werden.

\subsection{Demo und Test}

\subsubsection{myMonster}
Das Frauenhofer Institut stellt eine VideoChat Anwendung bereit, die auf ein IMS als Nutzerverzeichniss zurückgreift.
Zum testen der einzelnen Services wurde der Client \lstinline{monster-0.9.25} verwendet.

Über IPv6 kann der Proxy nicht erreicht werden \lstinline{Netzwerk nicht erreichbar}.
In den logs des Proxys wird der Kontakt \lstinline{C: <0://2001:4>} angezeigt obwohl dort eine vollständige IPv6 Adresse stehen sollte.

Wird das gesamte Labor Netzwerk auf IPv4 Betrieb umgestellt, funktioniert sowohl das An- und Abmelden, als auch das verschicken einer Nachricht. Deaktiviert man ipsec in den Einstellungen des myMonster Clients, sieht man mit wireshark, dass diese Nachrichten als SIP-Notify versendet werden. Da im Labor keine Headsets oder Webcams verfügbar waren, konnte Videochat nicht getestet werden.

Mit dem älteren Client "monster-0.9.8" funktioniert An- und Abmeldung, jedoch nicht das versenden von Nachrichten. Die Ursache hierfür wurde nicht gefunden.

\subsubsection{Boghe IMS Client}
Ähnlich dem myMonster Client erlaubt der Boghe IMS Client v2.0.112.744\footnote{\url{https://code.google.com/p/boghe/downloads/list}}
Nachrichten und Videochat, allerdings nur unter Windows. Ein Test mit vollständigem IPv6 war erfolgreich und zeigt, dass das IMS Betriebsbereit ist.

\subsection{Benutzung}

Ausser dem Hinzufügen neuer Teilnehmer muss das IMS System nicht mehr verändert werden. Neue Benutzer werden wie folgt angelegt:

Screenshot WebOberfläche
Parameter erklähren
\todo{schreiben}

\section{OpenIMS}
\label{sec:openims}



\section{Fazit und Lessions Learned}
\label{sec:openims_fazit}


\section{Einleitung}
\label{sec:Einleitung}


\section{Einleitung}
\label{sec:Einleitung}


\section{Einleitung}
\label{sec:Einleitung}






% Literaturverzeichnis ---------------------------------------------------------
%   Das Literaturverzeichnis wird aus der BibTeX-Datenbank "lit.bib"
%   erstellt.
% ------------------------------------------------------------------------------
% \nocite{*} % alles auflisten
\bibliographystyle{Bib/plaindin} % DIN-Stil des Literaturverzeichnisses
\bibliography{Bib/lit} % Aufruf: bibtex Bachelorarbeit



% Anhang -----------------------------------------------------------------------
%   Die Inhalte des Anhangs werden analog zu den Kapiteln inkludiert.
%   Dies geschieht in der Datei "Anhang.tex".
% ------------------------------------------------------------------------------
%\begin{appendix}
%    \clearpage
%    \pagenumbering{roman}
%    \chapter{Anhang}
%    \label{sec:Anhang}
%     Rand der Aufz?hlungen in Tabellen anpassen
%    \setdefaultleftmargin{1em}{}{}{}{}{}
%    \section{Signalisierungsprotkoll}

\ac{SIP} ist ein Signalisierungsprotkoll um eine Kommunikationssitzung zwischen zwei
oder mehreren Benutzern als auch Geräten auf- und abzubauen sowie zu steuern.
Die Signalisierung kann dabei in einem lokalen Netzwerk oder via Internet genutzt werden,
um die Gegenstellen von der initierten Verbindungsabsicht \zb für einen Datenaustausch
zu informieren.

Bild OSI-Schichtenmodell, welche Vorteile besitzt SIP gegeüber anderen Protokollen.
Gibt es \evtl Nachteile, welche?

Wie sieht eine SIP-Session aus auf- und abbau sowie der eigentliche Datenaustausch.
Wieso brauch ich eigentlich SIP, gibt es keine andere Möglichkeit?
SIP klärt lediglich die Kommunikationsmodalitäten, für die Datenübertragung müssen 
auf andere Protokolle zurückgegriffen werden.

Bild typische SIP-Kommunikation

Bestandteile einer typischen SIP-Infrastruktur sind:
- SIP Clients
- SIP Registrar Server
- SIP Proxy Server
- SIP Redirect Server
TODO: Erläuterung der einzelnen Geräte

PROBLEM IPv4: Durch die eingeschränkten IP-Adressen werden Lokale Netze hinter NAT-Routern 
aufgebaut, die dann über den NAT-Router eine Verbindung ins Internet bekommen.
Dadurch können SIP-Clients hinter NAT-Routern nicht direkt erreicht werden.
Abhilfe durch STUN, UPnP, usw ...
Oder besser, Verwendung von IPv6.

SIP-Nachrichten können für die Heimautomatisierung genutzt werden.
SIP MESSAGE PUT oder Get Aktionen (RFC 3261)
SIP SUBSCRIBE und NOTIFY für spezielle Event Benachrichtigungen (RFC 3265)


\section{Open IP Multimedia Subsystem}
 
%\end{appendix}

% Index ------------------------------------------------------------------------
%   Zum Erstellen eines Index, die folgende Zeile auskommentieren.
% ------------------------------------------------------------------------------
%\printindex

\end{document}
