\section{Installation und Konfiguration}
\label{sec:openims_config}
\todo{Rechtschreibfehler in der ganzen Section}

\subsection{Die Komponenten}
Auf der fertig eingerichteten Infrastruktur wurden die verschiedenen IMS Komponenten nach den Schritten die im offiziellen Install Guide auf openimscore.org erklärt werden\footnote{\url{http://www.openimscore.org/installation_guide}} instaliert.
Da fertige Debian Pakete bereit gestellt wurden, entfielen die Schritte zwei und drei.

Die Komponenten Wurden wie folgt verteilt:

\subsubsection{Home Subscriber Server}
Der HSS ist auf der Maschiene \lstinline{hss} instaliert, ausserdem ein MySQL Server der für den Betrieb dess HSS benötigt wird und das neueste JDK.

\subsubsection{Weboberfläche}

Der HSS kann über eine Weboberfläche konfiguriert werden, welche auf dem Apache Tomcat läuft und auf eine weitere MySQL Datenbank erfordert.
Diese wurde ebenfalls auf der Maschine \lstinline{hss} installiert.

Wenn in der Datei  \lstinline{/opt/OpenIMSCore/FHoSS/config/hhss.properties} als \lstinline{isten-Interface} eine IPv6 Adresse eingetragen ist, wird eine \lstinline{javax.management.MalformedObjectNameException(INVALID CHARACTERS ":")} geworfen. Um dieses Problem zu umgehen, wurde die IPv4 Adresse eingetragen und eine neuer DNS Eintrag \lstinline{hss4} erstellt der auf die IPv4 Adresse des \lstinline{hss} zeigt.
Das Webinterface ist unter der Adresse \url{http://hss4:8080} erreichbar. 
Benutzer für das WebInterface werden in der Datei \lstinline{/opt/OpenIMSCore/FHoSS/tomcat/conf/tomcat-users.xml} verwaltet. Die folgenden sind bisher vorhanden:
\begin{description}
\item [hss] mit Passwort \lstinline{hss} hat die Rolle \lstinline{Benutzer}
\item [hssAdmin] mit dem Passwort \lstinline{hss} hat die Rolle \lstinline{Admin}
\end{description}
\todo{stimmt das noch}

Das Tool zur Konfiguration des HSS, \lstinline{dpkg-reconfigure openimscore-fhoss} darf nicht mehr verwendet werden, da es die Resolv Einstellungen verändert und die Weboberfläche dann nicht mehr erreichbar ist.

\subsection{Interrogation-CSCF}
Der Interrogator wurde auf der Maschiene \lstinline{icscf} installiert. Er benötigt für den Betrieb einen MySQL Server der lokal installiert wurde.
\lstinline{dpkg-reconfigure openimscore-icscf} kann für weitere Konfiguration verwendet werden.

\subsection{Server-CSCF}
Die Serving Komponente wurde auf dem \lstinline{scscf} installiert.
\lstinline{dpkg-reconfigure openimscore-scscf} kann für weitere Konfiguration verwendet werden.

\subsection{Proxy-CSCF}
Ein Proxy ist auf dem  \lstinline{pcscf} installiert. dabei gab es nur einen kleineren Fehler, im Debian Pakte hat die Datei \lstinline{/usr/lib/ser/emerg_info.xml} gefehlt, diese wurde aus dem Upstream Repository kopiert.
Das Tool \lstinline{dpkg-reconfigure openimscore-pcscf} kann zur weiteren Konfiguration verwendet werden.

\subsection{Demo und Test}

\subsubsection{myMonster}
Das Frauenhofer Institut stellt eine VideoChat Anwendung bereit, die auf ein IMS als Nutzerverzeichniss zurückgreift.
Zum testen der einzelnen Services wurde der Client \lstinline{monster-0.9.25} verwendet.

Über IPv6 kann der Proxy nicht erreicht werden \lstinline{Netzwerk nicht erreichbar}.
In den logs des Proxys wird der Kontakt \lstinline{C: <0://2001:4>} angezeigt obwohl dort eine vollständige IPv6 Adresse stehen sollte.

Wird das gesamte Labor Netzwerk auf IPv4 Betrieb umgestellt, funktioniert sowohl das An- und Abmelden, als auch das verschicken einer Nachricht. Deaktiviert man ipsec in den Einstellungen des myMonster Clients, sieht man mit wireshark, dass diese Nachrichten als SIP-Notify versendet werden. Da im Labor keine Headsets oder Webcams verfügbar waren, konnte Videochat nicht getestet werden.

Mit dem älteren Client "monster-0.9.8" funktioniert An- und Abmeldung, jedoch nicht das versenden von Nachrichten. Die Ursache hierfür wurde nicht gefunden.

\subsubsection{Boghe IMS Client}
Ähnlich dem myMonster Client erlaubt der Boghe IMS Client v2.0.112.744\footnote{\url{https://code.google.com/p/boghe/downloads/list}}
Nachrichten und Videochat, allerdings nur unter Windows. Ein Test mit vollständigem IPv6 war erfolgreich und zeigt, dass das IMS Betriebsbereit ist.

\subsection{Benutzung}

Ausser dem Hinzufügen neuer Teilnehmer muss das IMS System nicht mehr verändert werden. Neue Benutzer werden wie folgt angelegt:

Screenshot WebOberfläche
Parameter erklähren
\todo{schreiben}
