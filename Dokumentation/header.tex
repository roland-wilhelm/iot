\documentclass[12pt,a4paper,final]{scrartcl}%article}
\usepackage[utf8]{inputenc}
\usepackage[T1]{fontenc} % T1 Encodierung fr europische Zeichenstze
\setcounter{secnumdepth}{4}
\setcounter{tocdepth}{3}
\usepackage{ngerman} % neue deutsche Rechtschreibung
\usepackage{graphicx}
%\usepackage{grffile}
\usepackage{tabularx}
\usepackage{amsmath} % fr mathematische Formeln
\usepackage{amssymb} % fr mathematische Symbole
\usepackage{textcomp,latexsym} % zustzliche Symbole
\usepackage{float} % fr Flieumgebungen
\usepackage[left=2.5cm,right=2.5cm,top=2cm,bottom=2cm,includeheadfoot]{geometry}
\usepackage{url}
\usepackage{fancybox}
\usepackage[pdftex,pdfpagelabels,plainpages=true,colorlinks=true,hypertexnames=false,pdfborder=false]{hyperref}
%\usepackage{lscape}
%\usepackage{totpages} %Total pages-> \ref{TotPages}
\usepackage{color}
\definecolor{Mygrey}{gray}{0.90}
\usepackage[figurewithin=section]{caption}

%Konfigutation für Codelisting
%see: http://stackoverflow.com/questions/741985
\usepackage{listings} \lstset{numbers=left, numberstyle=\tiny, numbersep=5pt} 
\lstloadlanguages{C,Java,sh}
\lstset{
	basicstyle=\footnotesize\ttfamily, % Standardschrift
	numbers=left,               % Ort der Zeilennummern
	numberstyle=\tiny,          % Stil der Zeilennummern
	numbersep=5pt,              % Abstand der Nummern zum Text
	tabsize=2,                  % Groesse von Tabs
	extendedchars=true,         %
	breaklines=true,            % Zeilen werden Umgebrochen
	stringstyle=\ttfamily, % Farbe der String
	showspaces=false,           % Leerzeichen anzeigen ?
	showtabs=false,             % Tabs anzeigen ?
	xleftmargin= 17pt,
	framexleftmargin=17pt,
	framexrightmargin=5pt,
	framexbottommargin=4pt,
	showstringspaces=false      % Leerzeichen in Strings anzeigen ?        
}
\DeclareCaptionFont{white}{\color{white}}
\DeclareCaptionFormat{listing}{\colorbox[cmyk]{0.43, 0.35, 0.35,0.01}{\parbox{\textwidth}{\hspace{15pt}#1#2#3}}}
\captionsetup[lstlisting]{format=listing,labelfont=white,textfont=white, singlelinecheck=false, margin=0pt, font={bf,footnotesize}}	


\usepackage[intoc]{nomencl}
\usepackage{nomencl}
\let\abbrev\nomenclatures
\renewcommand{\nomname}{list of abbreviations}
\setlength{\nomlabelwidth}{.25\hsize}
\renewcommand{\nomlabel}[1]{#1 \dotfill}
\setlength{\nomitemsep}{-\parsep}
\renewcommand{\figurename}{figure}

%\usepackage{thebibliography}

\usepackage{chngcntr} %ändern der counter
\counterwithin{figure}{section}
\counterwithin{table}{section}
%\counterwithin{equation}{section} 

\restylefloat{figure} % Platzierung H verschiebt nicht
\restylefloat{table} % Platzierung H verschiebt nicht\usepackage{algpseudocode}

\makeatletter
\renewcommand\l@figure{\@dottedtocline{1}{1.5em}{2.8em}}
\makeatother

\parindent=0pt
\parskip=0pt

\sloppy %blocksatz auch wenn größere Wortabstände in einer Zeile sein werden ;-)

\newcommand{\utsection}[2]{%Section mit Untertitel (ut)
    \section[#1]{#1\newline\normalfont\small\textit{Von: #2}}
}
\newcommand{\utsubsection}[2]{%Subsection mit Untertitel (ut)
    \subsection[#1]{#1\newline\normalfont\small\textit{Von: #2}}
}

\setlength{\parskip}{1.0ex plus0.4ex minus 0.4ex} % absatzabstand
%\renewcommand{\baselinestretch}{1.5}\normalsize %1,5 Zeilenabstand zieht das inhaltsverzeichnis so lang. wenn ihr den unbedingt wollt, dann definiert ihn nach dem Inhaltsverzeichnis, bitte!

% Abstnde fr Text und berschriften
\newcommand{\settocdepth}[1]{\addtocontents{toc}{\protect\setcounter{tocdepth}{#1}}}
\newcommand{\code}[1]{\ttfamily{\normalsize{\sloppy{#1}}}\normalfont}

% Abstand zweier Listenelemente kleiner
%\setlength{\itemsep}{0ex plus0.1ex}

% Kopf- und Fuzeile
\usepackage{fancyhdr}
\pagestyle{fancy}
\fancyhf{}
\fancypagestyle{plain} % alle Seiten im Fancy-Format

\fancyhead[C]{\nouppercase{\leftmark}} %Kopfzeile mittig
\fancyfoot[C]{\thepage} %Fuzeile mittig
\renewcommand{\headrulewidth}{0.5pt} %Linie oben
\renewcommand{\footrulewidth}{0.5pt} %Linie unten

% manuelle Silbentrennung
%\hyphenation{ber-wach-ungs-Funk-tio-na-li-tt}
