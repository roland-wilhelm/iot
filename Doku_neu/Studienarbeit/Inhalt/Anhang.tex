\section{Signalisierungsprotkoll}

\ac{SIP} ist ein Signalisierungsprotkoll um eine Kommunikationssitzung zwischen zwei
oder mehreren Benutzern als auch Geräten auf- und abzubauen sowie zu steuern.
Die Signalisierung kann dabei in einem lokalen Netzwerk oder via Internet genutzt werden,
um die Gegenstellen von der initierten Verbindungsabsicht \zb für einen Datenaustausch
zu informieren.

Bild OSI-Schichtenmodell, welche Vorteile besitzt SIP gegeüber anderen Protokollen.
Gibt es \evtl Nachteile, welche?

Wie sieht eine SIP-Session aus auf- und abbau sowie der eigentliche Datenaustausch.
Wieso brauch ich eigentlich SIP, gibt es keine andere Möglichkeit?
SIP klärt lediglich die Kommunikationsmodalitäten, für die Datenübertragung müssen 
auf andere Protokolle zurückgegriffen werden.

Bild typische SIP-Kommunikation

Bestandteile einer typischen SIP-Infrastruktur sind:
- SIP Clients
- SIP Registrar Server
- SIP Proxy Server
- SIP Redirect Server
TODO: Erläuterung der einzelnen Geräte

PROBLEM IPv4: Durch die eingeschränkten IP-Adressen werden Lokale Netze hinter NAT-Routern 
aufgebaut, die dann über den NAT-Router eine Verbindung ins Internet bekommen.
Dadurch können SIP-Clients hinter NAT-Routern nicht direkt erreicht werden.
Abhilfe durch STUN, UPnP, usw ...
Oder besser, Verwendung von IPv6.

SIP-Nachrichten können für die Heimautomatisierung genutzt werden.
SIP MESSAGE PUT oder Get Aktionen (RFC 3261)
SIP SUBSCRIBE und NOTIFY für spezielle Event Benachrichtigungen (RFC 3265)


\section{Open IP Multimedia Subsystem}
 